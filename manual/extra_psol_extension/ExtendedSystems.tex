\documentclass[11pt]{scrartcl}
\pdfoutput=1
\usepackage[scaled=0.9]{helvet}
\usepackage[scaled=0.9]{beramono}
\usepackage{amsmath,graphicx,upquote}
\usepackage{gensymb,paralist}
%\usepackage{mathpazo}
%\usepackage{eulervm}
%\usepackage[notref,notcite]{showkeys}
\usepackage[charter]{mathdesign}
\usepackage{color,listings,calc,url}
\typearea{11}
\usepackage[pdftex,colorlinks]{hyperref}
\definecolor{darkblue}{cmyk}{1,0,0,0.8}
\definecolor{darkred}{cmyk}{0,1,0,0.7}
\hypersetup{anchorcolor=black,
  citecolor=darkblue, filecolor=darkblue,
  menucolor=darkblue,pagecolor=darkblue,urlcolor=darkblue,linkcolor=darkblue}
%\renewcommand{\floor}{\operatorname{floor}}
\newcommand{\mt}[1]{\mathrm{#1}}
\newcommand{\id}{\mt{I}}
\newcommand{\matlab}{\texttt{Matlab}}
\renewcommand{\i}{\mt{i}}
\renewcommand{\d}{\mathop{}\!\mathrm{d}}
\renewcommand{\epsilon}{\varepsilon}
\newcommand{\sign}{\operatorname{sign}}
\newcommand{\atant}{\blist{atan2}}
\providecommand{\e}{\mt{e}}
\newcommand{\re}{\mt{Re}}
\newcommand{\im}{\mt{Im}}
\newcommand{\nbc}{n_\mt{bc}}
\newcommand{\gbc}{g_\mt{bc}}
\newcommand{\gic}{g_\mt{ic}}
\newcommand{\nic}{n_\mt{ic}}
\newcommand{\R}{\mathbb{R}}
\usepackage{microtype}

\definecolor{var}{rgb}{0,0.25,0.25}
\definecolor{comment}{rgb}{0,0.5,0}
\definecolor{kw}{rgb}{0,0,0.5}
\definecolor{str}{rgb}{0.5,0,0}
\newcommand{\mlvar}[1]{\lstinline[keywordstyle=\color{var}]!#1!}
\newcommand{\blist}[1]{\mbox{\lstinline!#1!}}
\newlength{\tabw}
\lstset{language=Matlab,%
  basicstyle={\ttfamily},%
  commentstyle=\color{comment},%
  stringstyle=\color{str},%
  keywordstyle=\color{kw},%
  identifierstyle=\color{var},%
  upquote=true,%
  deletekeywords={beta,gamma}%
}

\title{Extended systems for 
  Delay-differential equations as 
  implemented in the extensions to  DDE BifTool}
\author{Jan Sieber}\date{}
\begin{document}
\maketitle
\tableofcontents
\begin{abstract}
  \noindent
  \textbf{\textsf{Abstract}}\quad This document lists the extended
  systems defining bifurcations of periodic orbits in delay
  differential equations (DDEs) with finitely many point delays. The
  delays are permitted to be system parameters or dependent on the
  state at arbitrary points in the history. All systems are
  boundary-value problems with periodic boundary conditions.

  The proposed systems have been implemented as extensions to
  DDE-BifTool, a numerical bifurcation analysis package for DDEs for
  Matlab or Octave. The document contains also practical details of
  the implementation, pointing to relevant routines.
\end{abstract}


\section{Background}
\label{sec:background}
The references \cite{MDO09,G00,D07,DCFKSW98} give background
information on the defining systems for bifurcations of periodic
orbits in ODEs.  The systems defining the periodic-orbit bifurcations
for constant or state-dependent delay can be written as periodic
boundary-value problems (BVPs) of DDEs that can be tracked using the
\blist{'psol'} routines of
DDE-BifTool\footnote{\url{http://twr.cs.kuleuven.be/research/software/delay/ddebiftool.shtml}}. The
theory for bifurcations of periodic orbits in DDEs with constant delay
is covered in textbooks \cite{S89,HL93,DGLW95}. Details of the
algorithms and theory behind DDE-BifTool are discussed in
\cite{ELS01,ELR02,SER02,RS07,VLR08}. The review \cite{RS07} also
refers to
\texttt{knut}\footnote{\url{http://gitorious.org/knut/pages/Home}}, an
alternative C++ package with a stand-alone user interface. The review
\cite{HKWW06} lists difficulties and open problems occuring with DDEs
with state-dependent delays. However, periodic BVPs can be reduced to
smooth finite-dimensional systems of algebraic equations
\cite{S12}. This means that the general bifurcation theory for DDEs
with state-dependent delay works as expected as far as the Hopf
bifurcation, regular branches of periodic orbits, the period doubling
bifurcation and Arnol'd tongues branching off at resonant points at a
torus bifurcation are concerned.

In the following all DDEs are assumed to have periodic boundary
conditions on the interval $[0,1]$.

\section{Folds of periodic orbits --- constant delays}
\label{sec:ext:fold}
In DDE-BifTool the system defining a periodic orbit is the periodic
DDE with $n_\tau$ delays
\begin{align}
  \label{eq:po:dde}
  \dot x(t)&=T\cdot f\left(x(t),x(t-\tau_j/T)_{j=1}^{n_\tau},p\right)\mbox{,}\\
  \label{eq:po:phas}
  0&=\int_0^1\dot x_\mathrm{ref}(t)^Tx(t)\d t
\end{align}
where the unknowns are $x\in C_p^n:=C_\mathrm{per}([0,1];R^n)$ (space
of periodic functions on $[0,1]$) and $T$. The criterion for a fold is
that this system has a simple singularity, that is, its linearization
has a simple nullvector $(v,\beta)\in C_p^n\times\R$. This nullverctor
satisfies the linear system of equations:
\begin{align}
  \label{eq:fold:dde}
  \dot v(t)=&T\cdot
  \left[\partial_1f(t)v(t)+\sum_{k=1}^{n_\tau}\partial_{k+1}f(t)v(t-\tau_k/T)\right]+\\
  \label{eq:fold:beta}
  &\beta\cdot\left[f(t)+
    \sum_{k=1}^{n_\tau}\partial_{k+1}f(t)\dot x(t-\tau_k/T)\frac{\tau_k}{T}\right]\mbox{,}\\
  \label{eq:fold:phas0}
  0&=\int_0^1\dot x_\mathrm{ref}(t)^Tv(t)\d t
\end{align}
Note that the factor in front of $\beta$ (in brackets) in
\eqref{eq:fold:beta} is the derivative of the original DDE
\eqref{eq:po:dde} with respect to $T$. In \eqref{eq:fold:dde} and
\eqref{eq:fold:beta} the terms $f(t)$ and $\partial_kf(t)$ stand for
\begin{align}\label{eq:ftdef}
  f(t)&=f\left(x(t),x(t-\tau_j/T)_{j=1}^{n_\tau},p\right)\mbox{,}&
  \partial_kf(t)&=\partial_kf\left(x(t),x(t-\tau_j/T)_{j=1}^{n_\tau},p\right)\mbox{,}
\end{align}
respectively.

\paragraph{Practical considerations}
The phase condition \eqref{eq:po:phas} is hard-coded inside
DDE-BifTool. It cannot be replicated using user-defined system
conditions (\blist{funcs.sys_cond}) because DDE-BifTool passes only
the current point to \blist{sys_cond}. Thus, we have to replace $\dot
x_\mathrm{ref}(t)^T$ in \eqref{eq:fold:phas0} with $\dot
x(t)^T$:
\begin{equation}
  \label{eq:fold:phas}
  0=\int_0^1\dot x(t)^Tv(t)\d t\mbox{.}
\end{equation}


Furthermore, DDE-BifTool does not give the user access to the
derivatives of the solution in past points (needed in
\eqref{eq:fold:beta}). Thus, one has to replace $\dot x(t-\tau_k/T)$ by
\begin{equation}
  \label{eq:fpastfold}
  \dot x(t-\tau_k/T)=f\left(x(t-\tau_k/T),
    x(t-(\tau_k+\tau_j)/T)_{j=1}^{n_\tau},p\right)\mbox{,}
\end{equation}
such that the extended system has delays consisting of all sums of
original delays. If we define $\tau_0=0$, then the
$(n_\tau+1)n_\tau/2$ different delays
\begin{displaymath}
  \tau_{j,k}=\tau_j+\tau_k\mbox{\quad ($k\geq j\geq0$),}
\end{displaymath}
show up in the extended system ($\tau_{0,0}=0$, so doesn't count).

\paragraph{Extended system}
\begin{table}[ht]
  \centering
  \begin{tabular}[t]{l@{\qquad}l}\hline\noalign{\medskip}
    \textbf{Position} in \blist{parameter} & \textbf{parameter}
    \\\noalign{\medskip}
    \blist{1:npar} & user system parameters\\
    \blist{npar+1} & $\beta$\\
    \blist{npar+2} & parameter enforced to be equal to period\\
    \blist{npar+2+((j-1)*ntau+(k-j)+1} & $\tau_j+\tau_k$ ($j=1,\ldots,n_\tau$, $k=j,\ldots,n_\tau$)\\\noalign{\medskip}\hline
  \end{tabular}
  \caption{Parameters of extended system for fold 
    (\blist{ntau=length(sys_tau())}$=n_\tau$)}
  \label{tab:foldpars}
\end{table}
The function variables (stored in \blist{point.profile}) are $(x,v)\in
C_p^{2n}=C_p^n\times C_p^n$. Inside \blist{pfuncs.sys_rhs} they are
accessed as \blist{x=xx(1:n,:)} and \blist{v=xx(n+1:end,:)}. The
system parameters of the extended system, assuming that the
user-defined system has \blist{npar} parameters, are given in
Table~\ref{tab:foldpars}.  The DDE defined by \blist{sys_rhs(xx,par)}
consists of (using the convention $\tau_{0}=0$ and
$\tau_{j,k}=\tau_j+\tau_k$ and implemented in \blist{sys_rhs_POfold})
\begin{align}
    \label{eq:po:impdde}
  \dot x(t)&= f\left(x(t-\tau_j)_{j=0}^{n_\tau},p\right)\mbox{,}\\
    \label{eq:fold:impdde}
    \dot v(t)&=\frac{\beta}{T}f(t)+\partial_1f(t)v(t)+\sum_{k=1}^{n_\tau}
    \partial_{k+1}f(t)\left[v(t-\tau_k)+\frac{\beta\tau_k}{T}f(t-\tau_k)\right]\mbox{, where}\\
    \nonumber
    f(t-\tau_k)&=f\left(x(t-\tau_{k,j})_{j=0}^{n_\tau},p\right)
\end{align}
The parameters in its argument \blist{par} are ordered as in
Table~\ref{tab:foldpars}. The columns of \blist{xx} have the ordering
\blist{xx(:,j*ntau+(k-j)+1)}$=x(t-\tau_{j,k})=x(t-\tau_j-\tau_k)$
($j=0,\ldots,n_\tau$, $k=j,\ldots,n_\tau$).

The additional $3+n_\tau(n_\tau+1)/2$ system conditions implemented in
\blist{sys_cond_POfold} are:
\begin{align}
  \label{eq:foldvtv1}
  0&=\int_0^1 v(t)^Tv(t)\d t+\beta^2-1\\
  0&=\int_0^1 \dot x(t)^Tv(t)\d t \label{eq:fold:phas1}\\
  0&=\blist{point.parameter(npar+2)-point.period} \label{eq:fold:tcopy}\\
  0&=\tau_{j,k}-\tau_j-\tau_k \mbox{\quad $j=1,\ldots,n_{\tau}$,
    $k=j,\ldots,n_{\tau}$}\label{eq:fold:taurel}
\end{align}
Incorporating a copy of the period as a parameter (and constraining it
by \eqref{eq:fold:tcopy}) is necessary because \blist{sys_rhs} does
not have direct access to the period of the orbit. An auxiliary
routine \blist{p_dot} is included in the folder
\texttt{ddebiftool} to compute scalar products between
\blist{'psol'} structures of the type \eqref{eq:fold:phas1} (not
directly called by the user but may be useful in other problems).

\paragraph{Initialization}
The Jacobian $J$ of the linearization of
\eqref{eq:po:dde}--\eqref{eq:po:phas} in a point \blist{x} can be
obtained with the DDE-BifTool routine \blist{psol_jac} (calling it
without free parameters: \blist{free_par=[]}).  If $J=USV^T$ is the
singular-value decomposition of $J$ then \blist{V(:,end)} is the
approximate nullvector of $J$. This column vector only has to be
reshaped and rescaled to achieve \eqref{eq:foldvtv1}:
\begin{lstlisting}
  v.profile=reshape(V(1:end-1,end),size(x.profile));
  v.parameter=V(end,end);
  vnorm=sqrt(p_dot(v,v,'free_par_ind',1));
  beta=v.parameter/vnorm;
  v.profile=v.profile/vnorm;
\end{lstlisting}
Then \blist{v.profile} is the approximation for $v(t)$ and
\blist{v.parameter} is the approximation for $\beta$.

\section{Period doublings and torus bifurcations --- constant
  delays}
\label{sec:ext:tr}
The extended system has the function components $x$, $u$ and $v$ (all
in $C_p^n$) and the additional parameters $\omega$ (angle of critical
Floquet multipiers) and $T_\mathrm{copy}$ (equal to the period of the
orbit). The linearized system is written in Floquet exponent
form. This means that, if $\exp(\i \omega\pi)$ is the Floquet multiplier
and $z$ the corresponding eigenfunction (satisfying
$z(1)=z(0)\exp(\i\omega\pi)$), then
\begin{displaymath}
  u(t)+\i v(t)=\exp(-\i\omega\pi t/T)\,z(t)
\end{displaymath}
such that $u$ and $v$ are periodic (and real). The linearized system
in Floquet exponent form (as implemented in \blist{sys_rhs_TorusBif}) is
\begin{align}
  \label{eq:tr:u}
  \dot u(t) &=\phantom{-}\frac{\pi\omega}{T}v(t)+
  \sum_{j=0}^{n_\tau}\partial_{j+1}f(t)\left[\phantom{-}\cos(\pi\omega\tau_j/T)u(t-\tau_j)+
  \sin(\pi\omega\tau_j/T)v(t-\tau_j)\right]\\
  \label{eq:tr:v}
  \dot v(t) &=-\frac{\pi\omega}{T}u(t)+
  \sum_{j=0}^{n_\tau}\partial_{j+1}f(t)\left[-\sin(\pi\omega\tau_j/T)u(t-\tau_j)+
  \cos(\pi\omega\tau_j/T)v(t-\tau_j)\right]\mbox{.}
\end{align}
Again, we used the convention $\tau_0=0$ and $\partial_kf(t)$ as
defined in \eqref{eq:ftdef}.  The additional system conditions (in
\blist{sys_cond_TorusBif}) are
\begin{align}
  0&=\int_0^1 u(t)^Tu(t)+v(t)^Tv(t)\d t-1\label{eq:vtvutu:norm}\\
  0&=\int_0^1 u(t)^Tv(t)\d t\label{eq:vtu:orth}\\
  0&=\blist{point.parameter(npar+2)-point.period} \label{eq:tr:tcopy}
\end{align}
where \blist{npar} is the number of parameters defined by the user.

\paragraph{Initialization}
The auxiliary routine \blist{mult_crit} extracts the critical Floquet
multiplier $\mu$ and the corresponding eigenfunction $z(t)$ for the
approximate bifurcation point \blist{x}. Then the periodic Floquet
mode $y(t)$ is obtained as
\begin{displaymath}
y(t)=y_r(t)+\i y_i(t)=\exp(-\log(\mu)t)\,z(t)\mbox{,}
\end{displaymath}
where $t\in[0,1]$ is already rescaled, and
$\omega=\blist{atan2(imag(mu),real(mu))}/\pi$. Finally, one has to
rescale $y$ to make it of unit length, and make its real and imaginary
part orthogonal. Define
% r=1/sqrt(utu+vtv);
% gamma=atan2(2*utv,vtv-utu)/2;
% qr=r*(upoint.profile*cos(gamma)-vpoint.profile*sin(gamma));
% qi=r*(upoint.profile*sin(gamma)+vpoint.profile*cos(gamma));
\begin{align*}
  r&=\sqrt{\langle y_r,y_r\rangle+\langle
    y_i,y_i\rangle}\mbox{,\quad and }\\
  \gamma&=\frac{1}{2}\blist{atan2}\,(2\langle
  y_r,y_i\rangle,\langle
  y_r,y_r\rangle-\langle y_i,y_i\rangle)\mbox{,\quad then}\\
  u(t)&=[y_r(t)\cos\gamma-y_i\sin\gamma]/r\mbox{,}\\
  v(t)&=[y_r(t)\sin\gamma+y_i\cos\gamma]/r
\end{align*}
satisfy $\langle u,v\rangle=0$ and $\langle u,u\rangle+\langle
v,v\rangle=1$, giving the profiles of $u$ and $v$, needed for the
extended system.

\section{Systems with state-dependent delays}
\label{sec:sd-dde}
For state-dependent delays, the system defining a periodic orbit is
the periodic DDE 
\begin{align}
  \label{eq:po:sddde}
  \dot x(t)&=T\cdot f\left(x_0,\ldots,x_{n_\tau},p\right)\mbox{,\quad where}\\
  x_k&=x(t-\tau_k(x_0,\ldots,x_{k-1},p)/T)\mbox{\quad ($\tau_0=0$,
    $k=0,\ldots,n_\tau$),}
  \label{eq:po:sdxk}
\end{align}
augmented with the phase condition \eqref{eq:po:phas} to define the
period $T$. In \eqref{eq:po:sdxk} the delays are functions
$\tau_k:\R^{n\times k}\times\R^{n_p}\mapsto\R$.
\subsection{The general variational problem}
\label{sec:vp}
 The variational problem of
\eqref{eq:po:sddde}---\eqref{eq:po:sdxk} with respect to $x$,
linearized in $(x(\cdot),T,p)$ is (using $(v(\cdot),\beta,q)$ as the
linearized deviations):
\begin{align}\allowdisplaybreaks
  \dot v(t)=&\beta
  f(t)+T\partial_{p}f(t)q+
  T\sum_{k=0}^{n_\tau}\partial_{x,k}f(t)[V_k+B_k+Q_k]\label{eq:vp}
  \mbox{,\qquad where}\\
  V_0=&v(t)\mbox{,}\qquad
  B_0=0\mbox{\quad ($\in\R^n)$,}\qquad
  Q_0=0\mbox{\quad ($\in\R^n)$,}\\
  V_k=&v(t-\tau_k(t)/T)-\frac{\dot
    x(t-\tau_k(t)/T)}{T}\sum_{j=0}^{k-1}\partial_{x,j}
  \tau_k(t)V_j\mbox{,\quad($k=1\ldots n_\tau$)}\\
  B_k=&\frac{\dot x(t-\tau_k(t)/T)}{T}\left[\frac{\tau_k(t)}{T}\beta-
    \sum_{j=0}^{k-1}\partial_{x,j} \tau_k(t)B_j\right]\mbox{,\quad($k=1\ldots n_\tau$)}\\
  Q_k=&\frac{\dot x(t-\tau_k(t)/T)}{T}\left[-\partial_p\tau_k(t)q-
    \sum_{j=0}^{k-1}\partial_{x,j}
    \tau_k(t)Q_j\right]\mbox{\quad($k=1\ldots
    n_\tau$).}\label{eq:vp:qk}
\end{align}
System~\eqref{eq:vp}---\eqref{eq:vp:qk} uses the notations (for
$k=0,\ldots,n_\tau$)
\begin{align}
    f(t)&=f\left(x_0,\ldots,x_{n_\tau},p\right)\mbox{,}&
    \tau_k(t)&=\tau_k\left(x_0,\ldots,x_{n_\tau},p\right)\mbox{,}\nonumber\\
    \partial_{x,k}f(t)&=\frac{\partial}{\partial
      x_k}f\left(x_0,\ldots,x_{n_\tau},p\right)\mbox{,}&
    \partial_{x,k}\tau_k(t)&=\frac{\partial}{\partial
      x_k}\tau_k\left(x_0,\ldots,x_{n_\tau},p\right)\mbox{,}\label{eq:vp:ft}\\
    \partial_pf(t)&=\frac{\partial}{\partial
      p}f\left(x_0,\ldots,x_{n_\tau},p\right)\mbox{,}&
    \partial_p\tau_k(t)&=\frac{\partial}{\partial
      p}\tau_k\left(x_0,\ldots,x_{n_\tau},p\right)\mbox{.}\nonumber
\end{align}
The vectors $V_k\in\R^n$, $B_k\in\R^n$ and $Q_k\in\R^n$ are the
partial derivatives of $x_k$ w.r.t.\ $x(\cdot)$, $T$ and $p$
respectively.

The variational problem (and, hence, the extended systems for periodic
orbit bifurcations) require the time derivatives of $x$ at times
$t-\tau_k(t)/T$. DDE-BifTool does not give the user-defined functions
access to past states of the time derivatives such that they have to
be determined via the right-hand side $f$.  We define the matrix
$\tau_{j,k}$ of delays iteratively ($j=0,\ldots,n_\tau$,
$k=0,\ldots,n_\tau$):
\begin{align}
  \label{eq:tau0jdef}
  \tau_{0,k}&=\tau_{k,0}=\tau_k(t)&&\mbox{for $k=0,\ldots,n_\tau$ as
     in
    \eqref{eq:vp:ft}}\\
  \tau_{j,k}&=\tau_{j,0}+\tau_k\left(x\left(t-\frac{\tau_{j,0}}{T}\right),
      \ldots,x\left(t-\frac{\tau_{j,k-1}}{T}\right),p\right)&&\mbox{for
        $k,j=1,\ldots,n_\tau$.}
\end{align}
Then we can define
\begin{equation}
  \label{eq:ftkdef}
  f_k(t)=f(x(t-\tau_{k,0}/T),\ldots,x(t-\tau_{k,n_\tau}/T),p)
\end{equation}
and replace $\dot x(t-\tau_k(t)/T)/T$ by $f_k(t)$ in
\eqref{eq:vp}---\eqref{eq:vp:qk} ($f_0(t)$ equals $f(t)$ in \eqref{eq:vp:ft}).

\paragraph{Practical considerations} Continuation of periodic orbit
bifurcations computes the delays $\tau_{j,k}$ and calls the right-hand
side $f$ at $(x(t-\tau_{k,0}),\ldots,x(t-\tau_{k,n_\tau}))$. The
function \mlvar{tauSD_ext_ind} creates an array \mlvar{xtau_ind} which
sorts the delays as follows: $\tau_{0,k}=\tau_{k,0}=\blist{tau(k)}$
for $1\leq k\leq n_\tau$,
$\tau_{j,k}=\blist{tau((j-1)*ntau+ntau+1+k)}$ for $j\geq1$ and
$k=1,\ldots,n_\tau$. Inside the array \mlvar{xx}, which is the first
argument of \blist{sys_rhs} and \blist{sys_tau},
$(x(t-\tau_{k,0}),\ldots,x(t-\tau_{k,n_\tau}))$ can be found as
\blist{xx(:,xtau_ind(k+1,:))}



\section{Fold of periodic orbits --- state-dependent delays}
\label{sec:sd:pofold}
The function \mlvar{sys_rhs_SD_POfold} extends the user-defined
nonlinear problem with the variational problem
\eqref{eq:vp}---\eqref{eq:vp:qk} restricted to $q=0$. As a
user-defined function it assumes that its argument $x(\cdot)$ has
period $T$ (thus, all delays and derivatives are re-scaled by
$T$). Repeating all definitions, \mlvar{sys_rhs_SD_POfold}
implements the following system:
\begin{equation}\label{eq:sd:pofold}
  \begin{aligned}
    \dot x(t)&=f\left(x_0,\ldots,x_{n_\tau},p\right)\mbox{,}\\
    \dot v(t)=&\frac{\beta}{T}f(t)+
    \sum_{k=0}^{n_\tau}\partial_{x,k}f(t)[V_k+B_k]
    \mbox{,\qquad where}\\
    V_0=&v(t)\mbox{,}\quad B_0=0\mbox{\quad ($\in\R^n)$,}\\
    V_k=&v(t-\tau_k(t))-f_k(t)\sum_{j=0}^{k-1}\partial_{x,j}
    \tau_k(t)V_j\mbox{,\quad($k=1\ldots n_\tau$)}\\
    B_k=&f_k(t)\left[\frac{\tau_k(t)}{T}\beta-
      \sum_{j=0}^{k-1}\partial_{x,j}
      \tau_k(t)B_j\right]\mbox{,\quad($k=1\ldots n_\tau$).}
  \end{aligned}
\end{equation}
System~\eqref{eq:sd:pofold} uses the notations
\begin{equation}\label{eq:sd:abbrevs}
  \begin{aligned}
    x_k&=x(t-\tau_k(x_0,\ldots,x_{k-1},p))&&\mbox{\quad ($\tau_0=0$),
       $k=0,\ldots,n_\tau$,}\\
    f(t)&=f\left(x_0,\ldots,x_{n_\tau},p\right)\mbox{,}&&\\
    \tau_k(t)&=\tau_k\left(x_0,\ldots,x_{n_\tau},p\right)&&\mbox{\quad for $k=0,\ldots,n_\tau$,}\\
    \partial_{x,k}f(t)&=\frac{\partial}{\partial
      x_k}f\left(x_0,\ldots,x_{n_\tau},p\right)
    &&\mbox{\quad for $k=0,\ldots,n_\tau$,}\\
    \partial_{x,k}\tau_k(t)&=\frac{\partial}{\partial
      x_k}\tau_k\left(x_0,\ldots,x_{n_\tau},p\right)
    &&\mbox{\quad for $k=0,\ldots,n_\tau$,}\\
    f_k(t)&=f(x(t-\tau_{k,0}),\ldots,x(t-\tau_{k,n_\tau}),p)
    &&\mbox{\quad for $k=0,\ldots,n_\tau$,}\\
    \tau_{0,k}&=\tau_{k,0}=\tau_k(t)&&\mbox{\quad for $k=0,\ldots,n_\tau$,}\\
    \tau_{j,k}&=\tau_{j,0}+\tau_k(x(t-\tau_{j,0}),
    \ldots,x(t-\tau_{j,k-1}),p)&&\mbox{\quad for
      $k,j=1,\ldots,n_\tau$.}
  \end{aligned}
\end{equation}
The rows of the argument \mlvar{xx} for \mlvar{sys_rhs_SD_POfold}
consist of $x(\cdot)$ and $v(\cdot)$. The parameter array is extended
by $\beta$ and $T_\mt{copy}$ (a copy of the period $T$, the relation
$T=T_\mt{copy}$ is enforced in \mlvar{sys_cond_POfold}). The two
differential equations in \eqref{eq:sd:pofold} are augmented with the
phase condition built into DDE-BifTool, and additional system
conditions. The additional $3+n_\tau(n_\tau+1)/2$ system conditions
implemented in \blist{sys_cond_POfold} are identical to the extra
conditions of the constant-delay case
\eqref{eq:foldvtv1}--\eqref{eq:fold:tcopy}:
\begin{align}
  \label{eq:sdfoldvtv1}
  0&=\int_0^1 v(t)^Tv(t)\d t+\beta^2-1\\
  0&=\int_0^1 \dot x(t)^Tv(t)\d t \label{eq:sdfold:phas1}\\
  0&=\blist{point.parameter(npar+2)-point.period}\mbox{.} 
  \label{eq:sdfold:tcopy}
\end{align}
As the additional delays are not parameters, condition
\eqref{eq:fold:taurel} is not present (controlled by the parameter
\mlvar{relations} of \blist{sys_cond_POfold} being empty).

\section{Period doublings and torus bifurcations --- state-dependent
  delays}
\label{sec:sd:tr}
The function \mlvar{sys_rhs_SD_TorusBif} extends the user-defined
nonlinear problem with a coupled pair of variational problems
\eqref{eq:vp}---\eqref{eq:vp:qk} restricted to $q=0$ and
$\beta=0$. The approach is identical to the extended system for
constant delays outlined in Section~\ref{sec:ext:tr}. However, it
includes the derivatives of the time-delays. The extended system for
torus bifurcations computes a pair of two functions $u(t)$ and $v(t)$
($u(t)+\i v(t)$ is the critical complex Floquet mode). Consequently,
the extended system for torus bifurcations with state-dependent delays
requires a pair of sequences $U_k$ and $V_k$ (instead of $V_k$ and
$B_k$ for the fold of periodic orbits). The full extended system for
torus bifurcations and period doublings is:
\begin{equation}
  \label{eq:sd:torus}
  \begin{aligned}
    \dot x(t)&=f\left(x_0,\ldots,x_{n_\tau},p\right)\mbox{,}\\
        \dot u(t)&= \frac{\pi\omega}{T}v(t)+
        \sum_{k=0}^{n_\tau}\partial_{x,k}f(t)U_k\mbox{,}\\
        \dot v(t)&= -\frac{\pi\omega}{T}u(t)+
        \sum_{k=0}^{n_\tau}\partial_{x,k}f(t)V_k\mbox{,\qquad where}\\
        U_0&=u(t)\mbox{,}\quad V_0=v(t)
        \mbox{,\quad and for $k=1\ldots,n_\tau$}\\
    %     \begin{bmatrix}
    %       U_k\\V_k
    %     \end{bmatrix}&=
    %     \begin{bmatrix}
    %     \cos\left(\frac{\pi\omega\tau_k(t)}{T}\right)\mbox{,}&
    %     \sin\left(\frac{\pi\omega\tau_k(t)}{T}\right)\\
    %     -\sin\left(\frac{\pi\omega\tau_k(t)}{T}\right)\mbox{,}&
    %     \cos\left(\frac{\pi\omega\tau_k(t)}{T}\right)
    %     \end{bmatrix}
    %     \begin{bmatrix}
    %       u(t-\tau_k(t))\\
    %       v(t-\tau_k(t))
    %     \end{bmatrix}-
    %     \begin{bmatrix}
    %       f_k(t)\sum_{j=0}^{k-1}\partial_{x,j}
    % \tau_k(t)U_j\\
    %       f_k(t)\sum_{j=0}^{k-1}\partial_{x,j}
    % \tau_k(t)V_j
    %     \end{bmatrix}
    %     -f_k(t)\sum_{j=0}^{k-1}\partial_{x,j}
    % \tau_k(t)
    % \begin{bmatrix}
    %   U_j\\V_j      
    % \end{bmatrix}\mbox{.}
        U_k&=\phantom{-}
        \cos\left(\frac{\pi\omega\tau_k(t)}{T}\right)u(t-\tau_k(t))
        +\sin\left(\frac{\pi\omega\tau_k(t)}{T}\right)v(t-\tau_k(t))
        -f_k(t)\sum_{j=0}^{k-1}\partial_{x,j}
    \tau_k(t)U_j\mbox{,}\\
        V_k&=-\sin\left(\frac{\pi\omega\tau_k(t)}{T}\right)u(t-\tau_k(t))+
        \cos\left(\frac{\pi\omega\tau_k(t)}{T}\right)v(t-\tau_k(t))
        -f_k(t)\sum_{j=0}^{k-1}\partial_{x,j}
    \tau_k(t)V_j\mbox{.}
  \end{aligned}
\end{equation}
System\eqref{eq:sd:torus} uses the same set \eqref{eq:sd:abbrevs} of
notations as the system for the fold of periodic orbits.  The
additional conditions and the initialization are identical to the constant-delay case in Section~\ref{sec:ext:tr}.

\bibliographystyle{unsrt} \bibliography{delay}
\appendix


\end{document}
