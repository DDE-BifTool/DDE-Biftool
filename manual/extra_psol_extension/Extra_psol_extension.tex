\documentclass[11pt]{scrartcl}
% $Id: Extra_psol_extension.tex 153 2017-02-20 21:45:32Z jansieber $
\usepackage[scaled=0.9]{helvet}
\usepackage[T1]{fontenc}
\usepackage[scaled=0.9]{beramono}
\usepackage{amsmath,graphicx,upquote}
\usepackage{gensymb,paralist}
\usepackage{mathpazo}
%\usepackage{eulervm}
%\usepackage[notref,notcite]{showkeys}
%\usepackage[charter]{mathdesign}
\usepackage{color,listings,calc,url}
\typearea{12}
\usepackage[pdftex,colorlinks]{hyperref}
\definecolor{darkblue}{cmyk}{1,0,0,0.8}
\definecolor{darkred}{cmyk}{0,1,0,0.7}
\hypersetup{anchorcolor=black,
  citecolor=darkblue, filecolor=darkblue,
  menucolor=darkblue,pagecolor=darkblue,urlcolor=darkblue,linkcolor=darkblue}
%\renewcommand{\floor}{\operatorname{floor}}
\newcommand{\mt}[1]{\mathrm{#1}}
\newcommand{\id}{\mt{I}}
\newcommand{\matlab}{\texttt{Matlab}}
\renewcommand{\i}{\mt{i}}
\renewcommand{\d}{\mathop{}\!\mathrm{d}}
\renewcommand{\epsilon}{\varepsilon}
\renewcommand{\phi}{\varphi}
\newcommand{\sign}{\operatorname{sign}}
\newcommand{\atant}{\blist{atan2}}
\providecommand{\e}{\mt{e}}
\newcommand{\re}{\mt{Re}}
\newcommand{\im}{\mt{Im}}
\newcommand{\nbc}{n_\mt{bc}}
\newcommand{\gbc}{g_\mt{bc}}
\newcommand{\gic}{g_\mt{ic}}
\newcommand{\nic}{n_\mt{ic}}
\newcommand{\R}{\mathbb{R}}
\usepackage{microtype}

\definecolor{var}{rgb}{0,0.25,0.25}
\definecolor{comment}{rgb}{0,0.5,0} \definecolor{kw}{rgb}{0,0,0.5}
\definecolor{str}{rgb}{0.5,0,0}
\newcommand{\mlvar}[1]{\lstinline[keywordstyle=\color{var}]!#1!}
\newcommand{\blist}[1]{\mbox{\lstinline!#1!}}  \newlength{\tabw}
\lstset{language=Matlab,%
  basicstyle={\ttfamily\small},%
  commentstyle=\color{comment},%
  stringstyle=\color{str},%
  keywordstyle=\color{kw},%
  identifierstyle=\color{var},%
  upquote=true,%
  deletekeywords={beta,gamma,mesh}%
} \title{Description of extensions \texttt{ddebiftool\_extra\_psol} and
  \texttt{ddebiftool\_extra\_rotsym}} \author{Jan Sieber}\date{\today}
\begin{document}
\maketitle
\noindent This addendum describes two extensions to DDE-BifTool
\cite{ELS01,ELR02,homoclinic,RS07,VLR08}, a bifurcation analysis
toolbox running in
\texttt{Matlab}\footnote{\url{http://www.mathworks.com}} or
\texttt{octave}\footnote{\url{http://www.gnu.org/software/octave}}. The
extension \texttt{ddebiftool\_extra\_psol} enables continuation of
periodic orbit bifurcations for delay-differential equations with
constant or state-dependent delay. The extension
\texttt{ddebiftool\_extra\_rotsym} enables continuation of relative
equilibria, relative periodic orbits and their local bifurcations in
systems with rotational symmetry. The extensions are implemented by
creating \emph{extended problems} and re-using the core DDE-BifTool
algorithms for periodic orbits.
\tableofcontents

\section{Overview}
\label{sec:quick}
To get up and running quickly, users familiar with DDE-BifTool should
look at the demo in folder \texttt{minimal\_demo} (script
\texttt{rundemo.m}). The results are pre-computed and stored in
\texttt{mat} files. Otherwise, please see the main manual and the
tutorial demos \texttt{neuron}, \texttt{sd\_demo} and \texttt{hom\_demo}
for an introduction to using basic DDE-BifTool.

\noindent This extension introduces three functions that can be called
by the user:
\begin{compactitem}
\item \blist{SetupPOfold} to initialize continuation of folds of periodic
  orbits,
\item \blist{SetupTorusBifurcation} to initialize continuation of
  torus bifurcations, and
\item \blist{SetupPeriodDoubling} to initialize
  continuation of period doublings (this is identical to torus
  bifurcations).
\end{compactitem}


\section{Functionality}
\label{sec:extra}

The folder \texttt{ddebiftool\_extra\_psol} contains functions that
implement tracking of
\begin{compactitem}
\item saddle-nodes (folds),
\item period doublings, and
\item torus bifurcations
\end{compactitem}
of periodic orbits in two parameters. This is done by creating an
\emph{extended} DDE from the user-provided right-hand side. The
periodic orbits of the extended DDEs are then continued using
DDE-BifTool's original routines for periodic orbits.  To access the
additional routines, the folder \texttt{ddebiftool\_extra\_psol} has
to be added to the path, for example:
\begin{lstlisting}
  addpath('../../ddebiftool');            % path to original DDE-BifTool
  addpath('../../ddebiftool_extra_psol'); % path to additional routines
\end{lstlisting}
(adapt to your own folder structure).
\subsection{Functions of interest in user scripts}
\label{sec:user}



\subsubsection{\blist{SetupPOfold} for folds of periodic orbits}
\label{fold}

\paragraph{Header}\
\begin{lstlisting}[basicstyle={\ttfamily\small}]
function [pfuncs,pbranch,suc]=SetupPOfold(funcs,branch,ind,varargin)
\end{lstlisting}

\paragraph{Brief description}
Sets up functions defining right-hand side (in \blist{pfuncs}) and the
initial part of the branch with the first two points (in
\blist{pbranch}) of a curve of folds of periodic orbits.

\paragraph{Inputs}
\begin{compactitem}
\item \blist{funcs}\quad Structure, containing the user-defined right-hand
  side, conditions, delays, etc., as created with \blist{set_funcs}.
\item \blist{branch}\quad Branch of periodic orbits along which the
  fold that the user intends to track was detected. This has to be a
  structure of the format expected and created by DDE-BifTool routines
  such as \blist{df_brnch} or \blist{br_contn}.
\item \blist{ind}\quad index of point along the branch that is close
  to the fold. The periodic orbit \blist{branch.point(ind)} will be
  used to create an initial guess.
\item Optional inputs (key-value pairs in arbitrary order, default in
  brackets):
  \begin{compactitem}
  \item \blist{'contpar'} (\blist{[]}, integer)\quad \emph{index
    of continuation parameters}. This argument can be used in two
    ways:
    \begin{compactitem}
    \item[o] empty (not given): the indices given in
      \blist{branch.parameters.free} will be used.
    \item[o]  array \blist{k} with more than a single integer:  \blist{k}
    will \emph{replace} the parameters in the field
    \blist{branch.parameters.free}.
    \end{compactitem}
  \item \blist{'dir'} (\blist{[]}, integer)\quad \emph{initial direction of
      branch}; index of parameter to be changed for second point along
    branch of folds. If empty, then only a single fold point will be
    computed.
  \item \blist{'step'} (\blist{1e-3}, real) how much the parameter is
    changed. For the second point on the branch the relation holds
    \begin{lstlisting}
pbranch.point(2).parameter(dir)=...
      pbranch.point(1).parameter(dir)+step;
    \end{lstlisting}
  \item \blist{'correc'} (\blist{true}, logical) Apply
    \blist{p_correc} to the initial points.
  \item \blist{'sys_deri'} (\blist{1e-4}), \blist{'sys_dtau'}
    (\blist{1e-4}), \blist{'hjac'} (\blist{1e-4}) deviations used in
    finite-difference formulas. If the user has provided the functions
    \mlvar{sys_deri} and \mlvar{sys_dtau} then these options will not
    be used for constant delays. For state-dependent delays the
    analytic Jacobian of the extended system is not yet implemented
    such that \blist{'hjac'} will be used in the finite-difference
    approximation of the Jacobian.
  \end{compactitem}
\end{compactitem}

\paragraph{Outputs}
If the output \blist{suc} is non-zero, then \blist{pfuncs} and \blist{pbranch} can be fed into \blist{br_contn}. For example,
\begin{lstlisting}
  pbrlong=br_contn(pfuncs,pbranch,100);
\end{lstlisting}
\begin{compactitem}
\item \blist{pfuncs} Structure defining right-hand side of the
  extended DDE for folds of periodic orbits. The structure has an
  additional field \blist{'get_comp'}, which can be used to extract
  the original solution components from the solution of the extended
  system (removing additional components and prameters, see example).
\item \blist{pbranch} A DDE-BifTool branch structure, containing the
  first point or the first two points (if desired by setting the
  optional argument \blist{'dir'}).
\item \blist{suc} (logical) indicates success of approximation (and
  correction if desrired) of initial points.
\end{compactitem}

\paragraph{Example}
See \texttt{minimal\_demo}, the Duffing oscillator with delayed feedback of the form
\begin{equation}\label{eq:duff}
%x''(t)+d*x'(t)+a*x(t)+x^3+b*[x(t)-x(t-tau)]=0, parameters [tau,a,b,d]
  \ddot x(t)+d \dot x(t)+a x(t)+x(t)^3+b[x(t)-x(t-\tau)]=0\mbox{,}
\end{equation}
with parameters $p=(\tau,a,b,d)$ (initially $p=(0, 0.5, 0.6, 0.05)$).
After continuation of the family of periodic orbits in $\tau$ ($p_1$)
one has a branch
{\small
\begin{verbatim}
>> per_orb
per_orb = 
       method: [1x1 struct]
    parameter: [1x1 struct]
        point: [1x173 struct]
\end{verbatim}
} with $173$ points, of which point $52$ is close to a fold
\begin{verbatim}
>> ind_fold
ind_fold =    52
>> per_orb.point(ind_fold).stability.mu(1:3)
ans =
   1.0000          
   0.9985          
  -0.0263 + 0.0219i
\end{verbatim}
such that one can call
\begin{lstlisting}
[pfuncs,pbranch]=SetupPOfold(funcs,per_orb,ind_fold,...
    'contpar',[3,1],'dir',3,'step',-1e-3);
pbranch=br_contn(pfuncs,pbranch,60);
\end{lstlisting}
The free parameters and the intial step in $p_3$ (as requested by
the arguments \blist{'dir'} and \blist{'step'}) are
\begin{verbatim}
>> pbranch.parameter.free
ans =     3     1     5     6     7
>> pbranch.point(2).parameter(3)
ans =    0.5990
>> pbranch.point(1).parameter(3)
ans =    0.6000
\end{verbatim}
Note that the extended system introduces the additional parameters
$(\beta,T_\mathrm{copy},\tau_\mathrm{ext})$ (see \cite{S13}), which
are stored at indices $5$ to $7$. The extensions are also visible when
looking at the points:
\begin{verbatim}
>> pbranch.point(1)
ans =    kind: 'psol'
    parameter: [1x7 double]
         mesh: [1x81 double]
       degree: 4
      profile: [4x81 double]
       period: 0.5386
\end{verbatim}
Use \blist{pfuncs.get_comp} to remove all additional components and parameters:
\begin{verbatim}
>> pfuncs.get_comp(pbranch.point(1),'solution')
ans =    kind: 'psol'
    parameter: [0.9343 0.5000 0.6000 0.0500]
         mesh: [1x81 double]
       degree: 4
      profile: [2x81 double]
       period: 0.5386
\end{verbatim}
The function \blist{pfuncs.get_comp} supports the strings
\begin{compactitem}
\item  \blist{'kind'}\quad returning \blist{'POfold'},
\item \blist{'solution'}\quad removing all extensions, returning a
  point of the same format as the original periodic orbits,
\item \blist{'nullvector'}\quad returning the extended components of
  \blist{point.profile} (in the example, rows $3$ and $4$), and the
  extension parameter $\beta$ (in the example in position $5$) as a
  \blist{'psol'} structure,
\item \blist{'delays'}\quad returning the additional parameter(s)
  $\tau_\mathrm{ext}$ (see \cite{S13} for their meaning).
\end{compactitem}
\paragraph{Warning}
For problems with constant delays the extended problem \mlvar{pfuncs}
introduces $n_\tau(n_\tau+1)/2$ additional delays as parameter
$\tau_\mt{ext}$. For problems with state-dependent delays the
additional delays do not show up as additional parameters. However,
the number of additional delays is $n_\tau(n_\tau+1)$. Since the
user-defined functions have to be called at all delay-shifted
collocation points the computations may become slow for problems with
many delays (particularly for state-dependent delay problems).

\subsubsection{\blist{SetupTorusBifurcation} for torus bifurcations}
\label{sec:torusbif}

\paragraph{Header}\
\begin{lstlisting}
function [trfuncs,trbranch,suc]=SetupTorusBifurcation(...
                                         funcs,branch,ind,varargin)
\end{lstlisting}

\paragraph{Brief description}
Sets up the right-hand side (in \blist{trfuncs}) and the first two
points (in \blist{trbranch}) of a curve of torus bifurcations or
period doublings of periodic orbits (which one depends on the Floquet
multipliers of \blist{branch.point(ind)}). The meaning of inputs and
outputs are the same as for \blist{SetupPOfold}, except that the
points on the resulting branch lie on a torus or period doubling
bifurcation curve. Also, the right-hand side of the extended DDE is
different from the corresponding output of \blist{SetupPOfold} (see
\cite{S13} for the extended system).

\paragraph{Example}
(See again \texttt{minimal\_demo}.) The point with index
\blist{ind_tr1} has two complex Floquet multipliers near the unit
circle: {\small
\begin{verbatim}
>> per_orb.point(ind_tr1).stability.mu(1:5)
ans =
   1.1643          
  -0.6292 + 0.8380i
  -0.6292 - 0.8380i
   1.0000          
   0.5968          
\end{verbatim}
} So, one can call
\begin{lstlisting}
[trfuncs,trbranch1]=SetupTorusBifurcation(funcs,per_orb,ind_tr1,...
    'contpar',[3,1],'dir',3,'step',-1e-3);
trbranch1=br_contn(trfuncs,trbranch1,50);
\end{lstlisting}
to continue the torus bifurcation in the two parameters $(\tau,b)$. An
interesting parameter to watch during torus continuation is $\omega$,
the first extended parameter, which is the angle of the complex
Floquet multipliers on the unit circle in multiples of $\pi$ (so, if
the Floquet multiplier is at $-1$, $\omega$ equals $1$ for period doublings).

\paragraph{Warning}
For problems with state-dependent delays the extended problem
\mlvar{pfuncs} introduces $n_\tau(n_\tau+1)$ additional delays. The
additional delays do not show up as additional parameters.  Since the
user-defined functions have to be called at all delay-shifted
collocation points the computations may become slow for problems with
many delays.

\subsubsection{\blist{SetupPeriodDoubling} for period doublings}
\label{sec:pd}
\paragraph{Header}\ 
\begin{lstlisting}
function [trfuncs,trbranch,suc]=SetupPeriodDoubling(...
                                         funcs,branch,ind,varargin)
\end{lstlisting}
This function is a wrapper, just calling
\blist{SetupTorusBifurcation}, to avoid the misleading name. The free
parameter $\omega$ equals unity for period doublings (corresponding to
a rotation by $\pi$).



\section{List of demos}\label{sec:demos}
\begin{compactitem}
\item \textbf{\texttt{minimal\_demo}}: Duffing oscillator with delayed
  feedback. The delay is a parameter \cite{YP09}. This demonstrates
  usage of \blist{set_funcs} and continuation of folds and torus
  bifurcations of periodic orbits in two parameters.
\item \textbf{\texttt{Mackey-Glass}}: Mackey-Glass equation
  \begin{displaymath}
    \dot x(t)=\beta
  \frac{x(t-\tau)}{1+x(t-\tau)^n}-\gamma x(t)\mbox{.}
\end{displaymath}
(\url{http://www.scholarpedia.org/article/Mackey-Glass_equation}). The
delay is a parameter. This demonstrates the continuation of period
doublings in two parameters. This demo has a switch
\blist{x_vectorize} at the top. Change it to \blist{false} to see the
effect of vectorization on speed.
% \item \textbf{\texttt{Bando}}: $N$ Cars on a ring road with delays due to
%   reaction time (model is taken from \cite{OWES10}). This demonstrates
%   the continuation of folds of periodic orbits in a $2N$-dimensional
%   model ($N=10$ by default).
\item \textbf{\texttt{nested}}: Demonstrates continuation of periodic
  orbits, computation of their stability and continuation of fold of
  periodic orbits in two parameters for a system with state-dependent delays and arbitrary levels of nesting. The default system is
  \begin{displaymath}
    \dot x(t)=x(t-p_1-x(t-p_1-x(t-p_1-x(t))))+p_2x^5\mbox{.}
  \end{displaymath}
\item \textbf{\texttt{humphriesetal}}: example studied in \cite{HDMU12},
%rhs=@(x,p)-p(5)*x(1,1,:)-p(1)*x(1,2,:)-p(2)*x(1,3,:);
%sys_ntau=@()2;
%tau=@(nr,x,p)p(2+nr)+p(6)*x(1,1,:);
%-gamma*x(t)-kappa1*x(t-a1-c*x(t))-kappa2*x(t-a2-c*x(t))
  \begin{displaymath}
    \dot x(t)=-\gamma x(t)-\kappa_1x(t-\alpha_1-cx(t))-
    \kappa_2x(t-\alpha_2-cx(t))\mbox{.}
  \end{displaymath}
  Demonstrates continuation of folds of periodic orbits and torus
  bifurcations for an equation with state-dependent delay.
% \item \textbf{\texttt{turning}}: model for turning with finite
%   stiffness of the tool in lateral direction \cite{IBS08}. The
%   implicit definition of the (state-dependent) delay is converted into
%   an ordinary differential equation for the delay using Baumgarte
%   reduction. The example deomnstrates the continuation of folds of
%   periodic orbits.
\end{compactitem}

\section{Relative equilibria and periodic orbits in systems with
  rotational symmetry}
\label{sec:rot}

As a showcase demonstrating how one can extend DDE-BifTool's
functionality, the subfolder \texttt{ddebiftool\_extra\_rotsym} and
the demo \texttt{rotsym\_demo} are included. The routines in
\texttt{ddebiftool\_extra\_rotsym} enable users to track relative
equilibria and periodic orbits of DDE systems with rotational symmetry
and constant delays. Specifically, if the right-hand side of the DDE
\begin{equation}\label{eq:rot}
  \dot x=f(x(t),x(t-\tau_1),\ldots,x(t-\tau_{n_\tau}),p)
\end{equation}
satisfies
\begin{displaymath}
  \exp(A\phi)f(x_0,x_1,\ldots,x_{n_\tau},p)=
  f(\exp(A\phi)x_0,\exp(A\phi)x_1,\ldots,
  \exp(A\phi)x_{n_\tau},p)
\end{displaymath}
for some anti-symmetric matrix $A$ ($A^T=-A$) and all $\phi\in\R$,
$x_0$,\ldots,$x_{n_\tau}\in\R^n$ and parameters $p$, then
\eqref{eq:rot} possesses two special types of solutions:
\begin{compactitem}
\item \emph{Relative equilibria} (or \emph{rotating waves}). These are
  periodic orbits of the the form
  \begin{equation}
    x(t)=\exp(A\rho t)x_0\mbox{.}\label{eq:rw}
\end{equation}
\item \emph{Relative periodic orbits} (or \emph{modulated
    waves}). These are invariant two-tori of the form
  \begin{equation}
    x(t)=\exp(A(\rho t+\phi))x_0(t)\label{eq:mw}
  \end{equation}
  for all $t\in\R$ and some $\phi\in\R$ and periodic function
  $x_0(\cdot)$ (that is, $x_0(t)=x_0(t+T)$ for all $t\in\R$ and some $T>0$).
\end{compactitem}
For DDE-BifTool to be useful for DDEs with rotational symmetry, it
has to be able to continue relative equilibria and their bifurcations
similar to classical equilibria, and relative periodic orbits and their
bifurcations similar to classical periodic orbits. The function
\blist{set_rotfuncs} in \texttt{ddebiftool\_extra\_rotsym} creates a
right-hand side \blist{rot_rhs} out of the user-given right-hand side
$f$ and the matrix $A$ (and $\exp(A)$ if provided) in rotating
coordinates. The rotation frequency $\rho$ is an additional unknown
that gets fixed with an additional phase condition (in
\blist{rot_cond}). The other functions in
\texttt{ddebiftool\_extra\_rotsym} create wrappers around their
classical counterparts found in \texttt{ddebiftool\_extra\_psol}. The
demo \texttt{rotsym\_demo} shows how to use the extensions for
rotational symmetry for the Lang-Kobayashi equations \cite{LK80}.



{\small\bibliographystyle{unsrt} \bibliography{manual}
}


\end{document}
