\documentclass[11pt]{scrartcl}
\pdfoutput=1
\usepackage[scaled=0.9]{helvet}
\usepackage[scaled=0.9]{beramono}
\usepackage{amsmath,graphicx,upquote}
\usepackage{gensymb,paralist}
%\usepackage{mathpazo}
%\usepackage{eulervm}
%\usepackage[notref,notcite]{showkeys}
\usepackage[charter]{mathdesign}
\usepackage{color,listings,calc,url}
\typearea{11}
\usepackage[pdftex,colorlinks]{hyperref}
\definecolor{darkblue}{cmyk}{1,0,0,0.8}
\definecolor{darkred}{cmyk}{0,1,0,0.7}
\hypersetup{anchorcolor=black,
  citecolor=darkblue, filecolor=darkblue,
  menucolor=darkblue,pagecolor=darkblue,urlcolor=darkblue,linkcolor=darkblue}
%\renewcommand{\floor}{\operatorname{floor}}
\newcommand{\mt}[1]{\mathrm{#1}}
\newcommand{\xeq}{x_\mathrm{eq}}
\newcommand{\id}{\mt{I}}
\newcommand{\matlab}{\texttt{Matlab}}
\renewcommand{\i}{\mt{i}}
\renewcommand{\d}{\mathop{}\!\mathrm{d}}
\renewcommand{\epsilon}{\varepsilon}
\newcommand{\sign}{\operatorname{sign}}
\newcommand{\atant}{\blist{atan2}}
\providecommand{\e}{\mt{e}}
\newcommand{\re}{\mt{Re}}
\newcommand{\im}{\mt{Im}}
\newcommand{\nbc}{n_\mt{bc}}
\newcommand{\gbc}{g_\mt{bc}}
\newcommand{\gic}{g_\mt{ic}}
\newcommand{\nic}{n_\mt{ic}}
\newcommand{\R}{\mathbb{R}}
\usepackage{microtype}

\definecolor{var}{rgb}{0,0.25,0.25}
\definecolor{comment}{rgb}{0,0.5,0}
\definecolor{kw}{rgb}{0,0,0.5}
\definecolor{str}{rgb}{0.5,0,0}
\newcommand{\mlvar}[1]{\lstinline[keywordstyle=\color{var}]!#1!}
\newcommand{\blist}[1]{\mbox{\lstinline!#1!}}
\newlength{\tabw}
\lstset{language=Matlab,%
  basicstyle={\ttfamily},%
  commentstyle=\color{comment},%
  stringstyle=\color{str},%
  keywordstyle=\color{kw},%
  identifierstyle=\color{var},%
  upquote=true,%
  deletekeywords={beta,gamma}%
}

\title{Additional Implementation details for the
  \blist{ddebiftool_extra_nmfm} extension} \author{Jan Sieber}\date{}
\begin{document}
\maketitle
\tableofcontents
\begin{abstract}
  \noindent
  \textbf{\textsf{Abstract}}\quad This document explains changes to
  the original \blist{ddebiftool_extra_nmfm} extension to enable the
  computation of normal forms for codimension-one and -two
  bifurcations in delay-differential equations with state-dependent
  delays.
\end{abstract}


\section{Background}
\label{sec:background}
The theses by Janssens, Wage and Bosschaert
\cite{Janssens:Thesis,W14,MMB15} give background information on the
general approach to computation of normal forms in delay-differential
equations (DDEs) of the form
\begin{equation}
  \label{eq:dde}
  \dot x(t)=f(x_t)\mbox{,\quad where $f:C([-\tau,0];\R^n)\mapsto\R^n$ 
    and $x_t(\theta)=x(t+\theta)$.}
\end{equation}
They develop a framework that permits them to transfer the methodology
developed for Ordinary Differential Equations (ODEs) by Kuznetsov \&
Govaerts, described, for example in \cite{Kuznetsov:Elements:2004,G00}
and implemented in MatCont \cite{DGK03}. Their framework is intended
for the case where $f$ is smooth as a map with arguments in
$C([-\tau,0];\R^n)$. This case is referred to as the \emph{constant delay
case}.

One ingredient to the computation of normal forms are $k$th-order
derivatives of $f$ in an equilibrium $x_*$
($x_*(\theta)=\operatorname{const}=\xeq$ and $f(x_*)=0$) with respect to
$k$ deviations $v_1$,\ldots,$v_k\in C([-\tau,0];\R^n)$, which are
$k$-linear maps in their arguments $v_j$ ($j=1,\ldots,k$) mapping into
$\R^n$, written as
\begin{equation}\label{eq:hod}
  \partial^kf(x_*)[v_1,\ldots,v_k]\mbox{.}
\end{equation}
In normal form computations, the deviations $v_j$ are all combinations
of eigenfunctions and generalized eigenfunctions of the linear
functional $\partial^1f(x_*)$ such that a deviation $v$ has the form
\begin{displaymath}
  v(\theta)=\sum_{\ell=1}^{m}q_{\ell}\theta^{p_{\ell}}\exp(\lambda_{\ell}\theta)\mbox{.}
\end{displaymath}
The coefficients $q_\ell$ are in $\R^n$, the powers $p_\ell$ are
non-negative integers and the numbers $\lambda_\ell$ are
complex. Deviations of this type are smooth such that $k$th-order
derivatives of the form \eqref{eq:hod} can also be computed for
functional satisfying only a milder regularity assumption. We use the
abbreviation $C^j$ for $C^j([-\tau,0];R^n)$, the space of $j$ times
continuously differentiable functions on $[-\tau,0]$ with values in
$\R^n$.

\paragraph{Mild smoothness} We call $f:C^0\mapsto\R^n$ mildly
differentiable $k$ times if $f$ is continuously differentiable $k$
times as a functional from $:C^k$ into $\R^n$, and if the following
extension condition holds: the map $\partial^jf(x)[v_1,\ldots,v_j]$
can be extended continuously to $C^j\times [C^{j-1}]^j$.

This means that the $j$th derivative of $f$ in $x$ may depend on all
derivatives of $x$ up to order $j$, but it may only depend on
derivatives of $v_1$, \ldots, $v_j$ up to order $j-1$.
\paragraph{Example}
Consider the functional $f(x)=p-x(-x(0))$, which would be the
right-hand side of the DDE with state-dependent delay $\dot
x(t)=p-x(t-x(t))$. The derivatives of $f$ in an arbitrary
$x\in C^j$ in a single direction $v\in C^j$ are:
\begin{equation}
\begin{aligned}
\partial^1f(x)v&=-x'(-x(0))v(0)-v(-x(0))\\
\partial^2f(x)[v,v]&=-x''(-x(0))v(0)^2+2v'(-x(0))v(0)\\
\partial^3f(x)[v,v,v]&=-x'''(-x(0))v(0)^3-3v''(-x(0))v(0)^2\\
\ldots\\
\partial^jf(x)[v]^j&=-x^{(j)}(-x(0))v(0)^j+(-1)^jjv^{(j-1)}(-x(0))v(0)^{j-1}\mbox{.}
\end{aligned}\label{eq:example:expansion}
\end{equation}
This simple example illustrates that $\partial^jf$ depends on
derivatives up to order $j$ of $x$ but only on derivatives up to order
$j-1$ of the deviation. This is true in general also for mixed
derivatives and, more generally, for functionals describing the
right-hand side of a DDE with state-dependent delay as implemented in
DDE-Biftool. The assumption on mild smoothness ensures that the
nonlinear problems encountered in continuation of families of periodic
solutions have solutions depending smoothly on parameters \cite{S12}.

Since the normal form analysis is performed only near equilibria, the
expansion of the right-hand side simplifies (all derivatives of $x$ in
\eqref{eq:example:expansion} are zero and $x$ equals $\xeq$ everywhere):
\begin{equation}
\begin{aligned}
\partial^1f(x)v&=-v(-\xeq)\\
\partial^2f(x)[v,v]&=2v'(-\xeq)v(0)\\
\partial^3f(x)[v,v,v]&=-3v''(-\xeq)v(0)^2\\
\ldots\\
\partial^jf(x)[v]^j&=(-1)^jjv^{(j-1)}(-\xeq)v(0)^{j-1}\mbox{.}
\end{aligned}\label{eq:example:equilibrium}
\end{equation}

\paragraph{Multi-linear forms applied to function segments}
The normal form computations require
\bibliographystyle{unsrt} \bibliography{nmfm}


\end{document}
